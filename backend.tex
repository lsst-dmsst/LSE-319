\section{Backend Services\label{sec:backend}}

The user-facing aspects of the LSST Science Platform will be built on top of a number of backend services that can roughly be divided into three categories: \textbf{database services}, \textbf{file services}, and \textbf{batch computing} services (bottom row of Figure~\ref{fig:layeredLSP}). The details of these services are described in the Data Management Design Document (\citeds{LDM-148}) and other associated documents; here, we only provide the high-level guidance as to the capabilities which these services will need to expose to the user (through the three aspects).

\subsection{Database Services}

Key LSST catalog data products (both for Level 1 and Level 2) will be stored in relational databases and made available for querying by users using the Structured Query Language (SQL) as well as Astronomical Data Query Language (ADQL). These products, and expectations surrounding their schemas, are further described in the \DPDD.

Besides serving the LSST catalogs, LSST databases will also provide a per-user database space allocation. Within this allocation, end-users (including groups) will be able to store selected or transformed subsets of the LSST dataset, or upload related datasets for joining to the LSST dataset. The size of this allocation is determined by the \SRD requirement to provide 10\% of total LSST computing and storage resources to LSST users.

\subsection{File Services}

LSST Science Platform will also provide a per-user file space allocation. End-users (including groups) may use this allocation to upload code, store selected or transformed subsets of the LSST dataset (e.g., images), and in general keep files needed to support their data analysis work. Note that some of this space may be provided in form of an object store, rather than a file system with POSIX-like semantics.

The size of this allocation is determined by the \SRD requirement to provide 10\% of total LSST computing and storage resources to LSST users.

\subsection{Batch Computing Services}

Analysis performed through the Portal, JupyterLab, and Web API will be served by a shared computing cluster. This cluster will be managed by a workload management system that ensures resources are allocated to individual users or groups based on pre-determined operational policies. The size of the batch computing resource is determined by the \SRD requirement to provide 10\% of total LSST computing and storage resources to LSST users.

The users will be able to launch jobs on the batch computing cluster primarily utilizing the APIs exposed through the JupyterLab and Web API aspects of the LSST Science Platform. Some functionality exposed through the Portal may potentially utilize the batch computing cluster as well.

%\subsection{Overall User Experience}
%
%At start of operations,
%this computing cluster will number 2,400 cores (approximately 18 TFLOPs),
%with 4 PB of file and 3 PB of database storage (numbers for the U.S.  DAC).
%These will be shared by all users, the number of whom we’re estimating in
%the low thousands.
%
%Not all users will be accessing the computing cluster concurrently; though
%difficult to predict with accuracy because of a lack of direct comparables,
%an estimate on order of a ~100 concurrent users is likely reasonable.  This
%would translate to typical allocations of ~20 cores per user, sufficient to
%enable preliminary end-user science analyses (working on catalogs, smaller
%number of images) and creation of some added-value (Level 3) data products.
%A good analogy is one of being given a server with a few TB of disk, few TB
%of database storage, that is co-located next to the LSST data, and with a
%chance to use tens to hundreds of cores for analysis (depending on system
%load).
%
%Note that for larger endeavors (e.g., pixel-level reprocessing of the entire LSST
%dataset), the users will be steered towards resources beyond the LSST DACs
%(e.g., national supercomputing centers, university computing centers, or the
%public cloud).
%
%\subsection{Integrated aspect: the User Workspace}